\documentclass[10pt]{beamer}

\usetheme[progressbar=frametitle]{metropolis}
\usecolortheme{aggie}

\usepackage{appendixnumberbeamer}
\usepackage{booktabs}
\usepackage[scale=2]{ccicons}

\usepackage{pgfplots}
\usepgfplotslibrary{dateplot}

\usepackage{xspace}
\newcommand{\themename}{\textbf{\textsc{metropolis}}\xspace}

\title{Adams solver for ODE}
\subtitle{Using Adams solver for 2- and 3- body problems}
% \date{\today}
\date{}
\author{Severin Denisenko}
\institute{Saint Petersburg State University}
% \titlegraphic{\hfill\includegraphics[height=1.5cm]{logo.pdf}}

\begin{document}

\maketitle

\begin{frame}{Table of contents}
  \setbeamertemplate{section in toc}[sections numbered]
  \tableofcontents%[hideallsubsections]
\end{frame}

\section[History]{History}

\begin{frame}{History}
	\begin{itemize}
		\item Method called after British astronomer John Couch Adams.
		\item Described in work Bashforth, F. and Adams, J. C. Theories of Capillary Action. London: Cambridge University Press, 1883.
		\item But apparently was invented long before in 1855.
		\item After this was completely forgotten.
        \item Apparently A. N. Krylov revived works of Adams and Bashforth.
	\end{itemize}
\end{frame}

\section[Formulation]{Formulation}

\begin{frame}{Formulation}
  For the problem:
  \begin{gather*}
      y'=f(x,y) \\
      y(x_0)=y_0
  \end{gather*}
  Extrapolation (or Adams-Bashforth) method:
  \begin{equation}
      y_{n+1} = y_n + h \sum_{\lambda=0}^{k} u_{\lambda}f(x_{n-\lambda}, y_{n-\lambda})
  \end{equation}
  Where $u_{\lambda}$ are calculated coefficients.
\end{frame}

\begin{frame}{Formulation}
  For the Adams extrapolation method:
  \begin{equation*}
      u_{\lambda} = \frac{(-1)^\lambda}{j!(k-\lambda-1)!} \int_{0}^{1} \prod_{ \substack{i=0 \\ i\neq\lambda}}^{k} (\nu+i) d \nu
  \end{equation*}
  The easiest way to calculate this integral is numerical.
\end{frame}

\begin{frame}{Order of approximation}
  The Adams extrapolation method of order $k$ have order of approximation $p = k$ (local errors is of order $O(h^{p+1})$).
  
  For 2-step Adams method:
  \begin{equation*}
    y_{n+2} = y_{n+1} + h((1-\lambda) f_{n+1} + \lambda f_n)
  \end{equation*}
  \begin{equation*}
    y_{n+2} = y(t_{n+1}) + h((1-\lambda)y'(t_{n+1})+\lambda y'(t_n))
  \end{equation*}
\end{frame}

\begin{frame}{Order of approximation}
  Using Taylor's theorem expand $y'(t_n)$ at $t_{n+1}$:
  \begin{equation*}
      y_{n+2} = y(t_{n+1}) + h((1-\lambda)y'(t_{n+1})+ \lambda (y'(t_{n+1}) - h y''(t_{n+1}) + O(h^2)))
  \end{equation*}
  \begin{equation*}
      y_{n+2} = y(t_{n+1}) + h y'(t_{n+1}) - \lambda h^2 y''(t_{n+1}) + O(h^3)
  \end{equation*}
  Since $\lambda = - 1/2$:
  \begin{equation*}
      y_{n+2} = y(t_{n+1}) + h y'(t_{n+1}) + \frac{1}{2} h^2 y''(t_{n+1}) + O(h^3)
  \end{equation*}
  End finaly:
  \begin{equation*}
    y(t_{n+2}) - y_{n+2} = O(h^3)
  \end{equation*}
\end{frame}

\section{Implementation}

\begin{frame}[fragile]{List of used software}
	\begin{itemize}
        \item MPFR is a multiple-precision floating-point computation library
		\item mpreal is a c++ library for convenient use of MPFR
		\item Boost::uBLAS is a wrapper of BLAS for convenient use of procedures
		\item C++ of 20 standard
		\item CMake as a build system
        \item clang-format for auto-formatting of code
        \item address sanitizer for debugging of memory issues
	\end{itemize}
\end{frame}

\section{2-body problem}

\begin{frame}[fragile]{Equation of motion}
  \begin{verbatim}
odes::ode_t ode = [](odes::real_t t,
                     odes::vector_t x)
-> odes::vector_t {
    odes::vector_t y(4);

    // pow is overrided for mpfr::mpreal
    odes::real_t z = pow(x[0] * x[0] + x[1] * x[1],
                         3.0 / 2.0);
    y[2] = -x[0] / z;
    y[3] = -x[1] / z;
    y[0] = x[2];
    y[1] = x[3];

    return y;
};
  \end{verbatim}
\end{frame}

\begin{frame}[fragile]{Equation of motion}
  \begin{verbatim}
odes::vector_t x0(4);
x0[0] = 1.0;
x0[1] = 0.0;
x0[2] = 0.0;
x0[3] = 0.5;
  \end{verbatim}
\end{frame}

\section{Solutions}

\begin{frame}{First order}
  \begin{figure}
    \begin{tikzpicture}
      \begin{axis}[
        mlineplot,
        width=\textwidth,
        height=7cm,
        legend style={at={(0.8,0.8)},anchor=west},
        xmin=-0.5, xmax=1.5,
        ymin=-0.7, ymax=0.7,
        ]

        \addplot table [y=y, x=x]{dt0dot1order1.dat};
        \addlegendentry{$dt = 0.1$}
        \addplot table [y=y, x=x, mark=none]{dt0dot01order1.dat};
        \addlegendentry{$dt = 0.01$}
        \addplot table [y=y, x=x, mark=none]{dt0dot001order1.dat};
        \addlegendentry{$dt = 0.001$}
        \addplot table [y=y, x=x, mark=none]{dt0dot0001order1.dat};
        \addlegendentry{$dt = 0.0001$}
        \addplot table [y=y, x=x, mark=none]{dt0dot00001order1.dat};
        \addlegendentry{$dt = 0.00001$}
      \end{axis}
    \end{tikzpicture}
  \end{figure}
\end{frame}

\begin{frame}{First order}
  \begin{table}
    \caption{First order method precision}
    \begin{tabular}{lr}
      \toprule
      dt & x on t=T\\
      \midrule
      0.1 & -0.702892101\\
      0.01 & 1.321601040\\
      0.001 & 1.237004522\\
      0.0001 & 1.031508929\\
      0.00001 & 1.003246538\\
      0.000001 & 1.000326558\\
      0.0000001 & 1.000032663\\
      \midrule
      exact & 1.0 \\
      \bottomrule
    \end{tabular}
  \end{table}
\end{frame}

\begin{frame}{Second order}
  \begin{figure}
    \begin{tikzpicture}
      \begin{axis}[
        mlineplot,
        width=\textwidth,
        height=7cm,
        legend style={at={(0.8,0.8)},anchor=west},
        xmin=-0.5, xmax=1.5,
        ymin=-0.7, ymax=0.7,
        ]

        \addplot table [y=y, x=x]{dt0dot1order2.dat};
        \addlegendentry{$dt = 0.1$}
        \addplot table [y=y, x=x, mark=none]{dt0dot01order2.dat};
        \addlegendentry{$dt = 0.01$}
        \addplot table [y=y, x=x, mark=none]{dt0dot001order2.dat};
        \addlegendentry{$dt = 0.001$}
      \end{axis}
    \end{tikzpicture}
  \end{figure}
\end{frame}

\begin{frame}{Second order}
  \begin{table}
    \caption{Second order method precision}
    \begin{tabular}{lr}
      \toprule
      dt & x on t=T\\
      \midrule
      0.1 & -0.28799841125620\\
      0.01 & 1.05254088205525\\
      0.001 & 1.00007653064497\\
      0.0001 & 1.00000007705246\\
      0.00001 & 1.00000000004135\\
      0.000001 & 0.99999999999949\\
      \midrule
      exact & 1.0 \\
      \bottomrule
    \end{tabular}
  \end{table}
\end{frame}

\begin{frame}{Third order}
  \begin{figure}
    \begin{tikzpicture}
      \begin{axis}[
        mlineplot,
        width=\textwidth,
        height=7cm,
        legend style={at={(0.8,0.8)},anchor=west},
        xmin=-0.5, xmax=1.5,
        ymin=-0.7, ymax=0.7,
        ]

        \addplot table [y=y, x=x]{dt0dot1order3.dat};
        \addlegendentry{$dt = 0.1$}
        \addplot table [y=y, x=x, mark=none]{dt0dot01order3.dat};
        \addlegendentry{$dt = 0.01$}
        \addplot table [y=y, x=x, mark=none]{dt0dot001order3.dat};
        \addlegendentry{$dt = 0.001$}
      \end{axis}
    \end{tikzpicture}
  \end{figure}
\end{frame}

\begin{frame}{Third order}
  \begin{table}
    \caption{Third order method precision}
    \begin{tabular}{lr}
      \toprule
      dt & x on t=T\\
      \midrule
      0.1 & -0.29895190046225\\
      0.01 & 0.88646978773574\\
      0.001 & 0.99988371313306\\
      0.0001 & 0.99999988404660\\
      0.00001 & 0.99999999984795\\
      0.000001 & 0.99999999999988\\
      0.0000001 & 1.00000000000005\\
      \midrule
      exact & 1.0 \\
      \bottomrule
    \end{tabular}
  \end{table}
\end{frame}

\begin{frame}{Fourth order (long double)}
  \begin{table}
    \caption{Fourth order method precision}
    \begin{tabular}{lr}
      \toprule
      dt & x on t=T\\
      \midrule
      0.1 & 0.88202009642157\\
      0.01 & 0.97785887807003\\
      0.001 & 0.99999932504978\\
      0.0001 & 0.99999999982612\\
      0.00001 & 0.99999999996357\\
      0.000001 & 0.99999999999983\\
      \midrule
      exact & 1.0 \\
      \bottomrule
    \end{tabular}
  \end{table}
\end{frame}

\begin{frame}{Fourth order (mpfr)}
  \begin{table}
    \caption{Fourth order method precision}
    \begin{tabular}{lr}
      \toprule
      dt & x on t=T\\
      \midrule
      0.1 & 0.88204993803128\\
      0.01 & 0.97785892721521\\
      0.001 & 0.99999932554845\\
      0.0001 & 0.99999999981791\\
      0.00001 & 0.99999999995995\\
      0.000001 & 1.00000000000000\\
      \midrule
      exact & 1.0 \\
      \bottomrule
    \end{tabular}
  \end{table}
\end{frame}

\begin{frame}{Runge-Kutta 4'th}
  \begin{figure}
    \begin{tikzpicture}
      \begin{axis}[
        mlineplot,
        width=\textwidth,
        height=7cm,
        legend style={at={(0.8,0.8)},anchor=west},
        xmin=-0.5, xmax=1.5,
        ymin=-0.7, ymax=0.7,
        ]

        \addplot table [y=y, x=x]{rk0dot1.dat};
        \addlegendentry{$dt = 0.1$}
        \addplot table [y=y, x=x, mark=none]{rk0dot01.dat};
        \addlegendentry{$dt = 0.01$}
      \end{axis}
    \end{tikzpicture}
  \end{figure}
\end{frame}

\begin{frame}{Runge-Kutta 4'th}
  \begin{table}
    \caption{Runge-Kutta 4'th method precision}
    \begin{tabular}{lr}
      \toprule
      dt & x on t=T\\
      \midrule
      0.1 & -4.39574629733722\\
      0.01 & 0.99988122872003\\
      0.001 & 0.99999957664443\\
      0.0001 & 0.99999999981837\\
      0.00001 & 0.99999999995897\\
      0.000001 & 1.00000000000000\\
      \midrule
      exact & 1.0 \\
      \bottomrule
    \end{tabular}
  \end{table}
\end{frame}

\section{Periodic 3-body solutions}

\begin{frame}{Three-body circled cross}
T = 24 $\pi$ \cite{psolnbody}
  \begin{figure}
    \begin{tikzpicture}
      \begin{axis}[
        mlineplot,
        width=\textwidth,
        height=7cm,
        xmin=-1.1, xmax=1.1,
        ymin=-1.1, ymax=1.1,
        axis equal
        ]

        \addplot table [y=y1, x=x1, mark=none]{circled_cross.dat};
        \addplot table [y=y2, x=x2, mark=none]{circled_cross.dat};
        \addplot table [y=y3, x=x3, mark=none]{circled_cross.dat};
      \end{axis}
    \end{tikzpicture}
  \end{figure}
\end{frame}

\begin{frame}{Three-body figure-eight}
T = 8 $\pi$
  \begin{figure}
    \begin{tikzpicture}
      \begin{axis}[
        mlineplot,
        width=\textwidth,
        height=7cm,
        xmin=-1.1, xmax=1.1,
        ymin=-1.1, ymax=1.1,
        axis equal
        ]

        \addplot table [y=y1, x=x1, mark=none]{figure_eight.dat};
        \addplot table [y=y2, x=x2, mark=none]{figure_eight.dat};
        \addplot table [y=y3, x=x3, mark=none]{figure_eight.dat};
      \end{axis}
    \end{tikzpicture}
  \end{figure}
\end{frame}

\appendix

\begin{frame}[allowframebreaks]{References}
  \bibliographystyle{abbrv}
  \bibliography{celmech}
\end{frame}

\end{document}
